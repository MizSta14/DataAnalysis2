\documentclass[letterpaper, 12pt]{article}


\usepackage{parskip,xspace}
\usepackage{amsmath,amsthm,amsfonts,amssymb}
\usepackage{mathrsfs} 
\usepackage{caption}
\usepackage{xcolor} 
\usepackage{geometry}
\usepackage{fancyhdr}
\usepackage{rotating}
\usepackage{multirow}
\usepackage{makecell}
\usepackage{ltxtable}
\usepackage{hyperref}
\usepackage{graphicx}
\usepackage{subfigure}
\usepackage{bm}
\usepackage[]{statrep}
\usepackage{enumerate}
\usepackage{subfigure}
\usepackage[toc,page]{appendix}

\graphicspath{{eps/}}


\newcommand{\ba}{$$\begin{aligned}}
\newcommand{\ea}{\end{aligned}$$}
\newcommand{\dx}{\mathrm{d}x}
\newcommand{\lma}{\left(\begin{matrix}}
\newcommand{\rma}{\end{matrix}\right)}




\pagestyle{fancy}
\lhead{Peng Shao 14221765}
\chead{}
\rhead{\bfseries STAT 8320 Spring 2015 Assignment 6}
\renewcommand{\headrulewidth}{0.4 pt}
\setlength{\parindent}{2em}

\begin{document}
\title{STAT 8320 Spring 2015 Assignment 6}
\author{Peng Shao 14221765}
\maketitle
\indent




$\blacktriangleright$ \textbf{1.\quad Solution.} 
(a). We can know that $n_1=16$, $n_2=11$ and $p=4$. Then to test the hypothesis
$$
H_0:\bm{\mu}_1-\bm{\mu}_2=\bm{0}\qquad\text{v.s.}\qquad H_A:\bm{\mu}_1-\bm{\mu}_2\not=\bm{0}
$$
The statistic is
$$
F=\frac{n_1+n_2-p-1}{(n_1+n_2-2)\cdot p}T^2=\frac{16+11-4-1}{(16+11-2)\cdot 4}\cdot16.5=3.49\sim f_{4,22}
$$
Because $F=3.49>f_{0.95,4,22}=2.816708$ with P-value=0.0237. We will reject the null hypothesis, so the mean profiles are significantly different.

(b). To test the parallel profiles, the hypotheses are
$$
H_0:\bm{C}(\bm{y}-\bm{z})=0\quad\text{v.s.}\quad H_A: \bm{C}(\bm{y}-\bm{z})\not=0
$$where $\bm{C}=\lma 1 &-1 &0 &0\\ 0 &1 &-1 &0\\ 0 &0 &1 &-1\rma$. 

The statistics are
$$
F=\frac{n_1+n_2-c-1}{(n_1+n_2-2)c}T^2=2.699\sim f_{3,23}
$$
Because $F=2.699<f_{0.95,3,23}=3.027998$ with P-value=0.06934349. We cannot reject the null hypothesis, so there is not significant interaction effect, i.e., the mean profiles are parallel.

(c). Since the mean profiles are parallel, then we want to test whether they are identical or not. Then the $\bm{C}$ matrix becomes 
$
\bm{C}=\lma 1 &1 &1 &1\rma
$.
This test should give us the same result of part (a).

We also want to test whether there is a significant time effect, i.e. each profile is horizontal respectively. Then the hypotheses are
$$
H_0:\bm{C\mu}_1=0\quad\text{v.s.}\quad H_A:\bm{C\mu}_1\not=0
$$
where $\bm{C}=\lma 1 &-1 &0 &0\\ 0 &1 &-1 &0\\ 0 &0 &1 &-1\rma$.




$\blacktriangleright$ \textbf{2.\quad Solution.} 
(a). Mauchly's test examines the form of the common covariance matrix. A spherical matrix has equal variances and covariances equal to zero. The common covariance matrix of the transformed within-subject variables must be spherical. So the purpose of Mauchly test is to help us decide which output to use. If we can use the univariate output, we may have more power to reject the null hypothesis in favor of the alternative hypothesis. However, the univariate approach is appropriate only when the sphericity assumption is not violated. If the sphericity assumption is violated, then in most situations you are better off staying with the multivariate output.

From the SAS output (Figure \ref{h6re7}), the second line shows that the Mauchly's criterion is 0.7343, $\chi^2$ value is 7.29 with degree of freedom 5, and the P-value is 0.1997$>$0.05. Thus, we cannot reject the null hypothesis of sphericity, that is, the sphericity assumption has not been violated. Then we should use univariate approach to make inferences, because it is more powerful to reject the null hypothesis.
\Listing[store=class,
         caption={Mauchly's Sphericity Test}]{h6re7}
         


(b). \begin{enumerate}[(i).]
\item To test the parallel profiles, the null hypothesis is
$$
H_0: \mu_{11}-\mu_{21}=\mu_{12}-\mu_{22}=\mu_{13}-\mu_{23}=\mu_{14}-\mu_{24}
$$
In Figure \ref{h6re9}, the Wilks' Lambda value is 0.7399, the F value is 2.70, the degree of freedom of numerator is 3, the degree of freedom of denominator is 23, and the P-value is 0.0696$>$0.05. So we cannot reject the null hypothesis, that is, the mean profiles are parallel, or we can say that there is no significant interaction effect.
\Listing[store=class,
         caption={Parallel Profiles Test}]{h6re9}
\item To test the parallel profiles, the null hypothesis is
$$
H_0: \mu_{11}=\mu_{12}=\mu_{13}=\mu_{14}
$$
In Figure \ref{h6re8}, the Wilks' Lambda value is 0.1948, the F value is 31.69, the degree of freedom of numerator is 3, the degree of freedom of denominator is 23, and the P-value is less than 0.0001$<$0.05. So we will reject the null hypothesis, that is, the mean profiles are not horizontal, or we can say that there is a significant age effect.
\Listing[store=class,
         caption={Horizontal Profiles Test}]{h6re8}
\item To test the parallel profiles, the null hypothesis is
$$
H_0: \bm{\mu}_1-\bm{\mu}_2=\bm{0}
$$
In Figure \ref{h6re10}, there is only one approach for this test. The F value is 9.29, the degree of freedom of numerator is 1, the degree of freedom of denominator is 25, and the P-value is 0.0054$<$0.05. So we will reject the null hypothesis, that is, the mean profiles are not identical, or we can say that there is a significant gender effect.
\Listing[store=class,
         caption={Coincident Profiles Test}]{h6re10}
\end{enumerate}


(c). From the SAS output (Figure \ref{h6re11}), we can see the F value of interaction is 2.36 with P-value=0.0797$>$0.05. So we cannot reject the null hypothesis of no interaction. Because of no interaction, it is appropriate to test the main effect. The F value of age is 35.35 with P-value less than 0.0001. So we can conclude that there is a significant effect of age.
\Listing[store=class,
         caption={Univariate Approach}]{h6re11}
         
(d). From the SAS output (Figure \ref{h6re12}), the Greenhouse-Geisser adjustment is 0.8672, and the Huynh-Feldt adjustment is 09769.
\Listing[store=class,
         caption={Univariate Approach Adjustment}]{h6re12}
         
(e). From the final SAS output of PROC MIXED (Figure \ref{h6re26}), we have the exactly same result that there is no significant interaction, but the gender effect and age effect are both significant  as univariate approach shows, especially the P-values of these three effects. This also support the result of Mauchly's test that the sphericity assumption is valid, which ensures the univariate approach is appropriate.
\Listing[store=class,
         caption={Result of PROC MIXED}]{h6re26}










$\blacktriangleright$ \textbf{3.\quad Solution.} 
(a). From the plot (Figure \ref{h6re}), firstly we can see that there are some intersections among the profiles, which means that there may be some significant interaction between reinforcement schedules and conditions. Then we can see that the profile of schedule 1 is parallel with the profile of schedule 2, and the difference is so small that they may behave in same way. Finally, the behavior of schedule 3 seems totally different from the other two's.
\Graphic[store=class, scale=0.7,
         caption={Plots of Mean Profiles}]{h6re}

(b). From the SAS output (Figure \ref{h6re44}), P-value of the test is 0.0119$<$0.05, so we can say that there is a significant schedule effect. The next step we should do is to find where the differences lie, so we will follow up with several post-hoc tests in part (c).
\Listing[store=class,
         caption={Overall Reinforcement Schedule Effect}]{h6re44}

(b). We will begin with the multivariate test of schedule 1 versus the average of schedule 2. The null hypothesis is
$$
H_0:\bm{LBM}=\bm{0}
$$
where $\bm{L}=\lma 1&-1&0\rma$, and
$$\bm{M}=\lma1&0&0\\-1&1&0\\0&-1&1\\0&0&-1\rma$$.
From the SAS output (Figure \ref{h6re49}), P-value of the test is 0.9979$>$0.05, so we cannot reject $H_0$ that schedule 1 and schedule 2 behave the same.
\Listing[store=class,
         caption={Overall Reinforcement Schedule Effect}]{h6re49}
         
We will begin with the multivariate test of schedule 3 versus the average of schedule 1 and 2. The null hypothesis is
$$
H_0:\bm{LBM}=\bm{0}
$$
where $\bm{L}=\lma -1 -1 +2\rma$, and $\bm{M}$ is same as above.

From the SAS output (Figure \ref{h6re51}), P-value of the test is 0.0030$<$0.05, so we will reject $H_0$ that schedule 3 behaves like the other 2 schedules under different conditions.
\Listing[store=class,
         caption={Overall Reinforcement Schedule Effect}]{h6re51}




%\Listing[store=class,
%  caption={Regression Analysis}]{resultl}

%\Graphic[store=class, scale=0.9,
%  caption={Graphs for Regression Analysis}]{resultg}

\end{document}
