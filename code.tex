\documentclass[letterpaper, 12pt]{article}


\usepackage{parskip,xspace}
\usepackage{amsmath,amsthm,amsfonts,amssymb} 
\usepackage{caption}
\usepackage{xcolor} 
\usepackage{geometry}
\usepackage{fancyhdr}
\usepackage{rotating}
\usepackage{multirow}
\usepackage{makecell}
\usepackage{ltxtable}
\usepackage{hyperref}
\usepackage{graphicx}
\usepackage{subfigure}
\usepackage{bm}
\usepackage[]{statrep}
\usepackage{enumerate}
\usepackage{subfigure}
\graphicspath{{eps/}}







\pagestyle{fancy}
\lhead{Peng Shao 14221765}
\chead{}
\rhead{\bfseries STAT 8320 Spring 2015 Assignment 2}
\renewcommand{\headrulewidth}{0.4 pt}
\setlength{\parindent}{2em}

\begin{document}
\title{STAT 8320 Spring 2015 Assignment 2}
\author{Peng Shao 14221765}
\maketitle
\indent



\begin{Sascode}[store=class]
libname da2 'C:\Users\psy6b\Desktop\8320 datasets'; 
ods graphics on; 
options ls=70 ps=35; 

data da2.h5q2;
infile 'C:\Users\psy6b\Desktop\8320 datasets\growthdata.dat';
input t @;
do i=1 to 6;
input y @;
output;
end;
run;

data aver;
set da2.h5q2;
by t;
retain count yave;
if first.t then do;
count=0;
yave=0;
end;
count+1;
yave+y;
if last.t then do;
y=yave/count;
i=7;
output;
end;
keep t i y;
run;

data da2.h5q2plot;
set da2.h5q2 aver;
run;

proc sort data=da2.h5q2plot;
by i;
run;

symbol interpol=join;
proc gplot data=da2.h5q2plot;
plot y*t=i;
run;
quit;

proc nlin data=da2.h5q2 hougaard noitprint method=newton;
parameters beta1=200
beta2=850
beta3=350;
e=exp(-(t-beta2)/beta3);
model y =beta1/(1+e);
run;
proc nlmixed data=da2.h5q2;
parameters beta1=200
beta2=850
beta3=350
resvar=40
varu=900;
e=exp(-(t-beta2)/beta3);
model y ~ normal((beta1+u)/(1+e), resvar);
random u ~ normal(0,varu) subject=i out=EBlups;
predict beta1/(1+e) out=pred;
predict (beta1+u)/(1+e) out=predB;
estimate 'Beta_3=350?' beta3-350;
ods output ParameterEstimates=estimates;
run;

proc sort data=pred;
by i t;
run;
proc sort data=predB;
by i t;
run;
data panelplot;
merge predB(rename=(pred=PredB)) pred;
by i t;
length type $20;
keep i t type resp;
type='measurement';
resp=y;
output;
type='cluster-specific';
resp=predb;
output;
type='population-average';
resp=pred;
output;
run;

proc sgpanel data=panelplot;
panelby i/spacing=5 rows=2 columns=3 novarname;
vline t/response=resp group=type;
colaxis fitpolicy=thin alternate;
rowaxis alternate;
run;

proc print data=eblups;
title 'Estimation of Random Effect';
var i Estimate tValue Probt;
run;

title;

/********************************************************/


/*To reading the data*/
data da2.h5q31;
infile 'C:\Users\psy6b\Desktop\8320 datasets\ssttornado532001.dat';
retain ss1-ss49;
array ss{49} ss1-ss49;
if _N_=1 then do; input ss1-ss49;end;
loc+1;
drop ss1-ss49;
do t=1 to 49;
sst=ss{t};
input torn @;
output;
end;
run;
data da2.h5q32;
infile 'C:\Users\psy6b\Desktop\8320 datasets\MOtornlatlon.dat';
loc+1;
input lat lon;
run;
proc sql;
create table da2.h5q3
as select * from da2.h5q31 as a, da2.h5q32 as b
where a.loc=b.loc;
run;
quit;

/*Fitting different models*/
proc genmod data=da2.h5q3;
class loc;
model torn = sst sst*loc / dist=poisson link=log;
output out=h5q3out1 resraw=Residual pred=Predicted lower=Lower upper=Upper;
proc glimmix data=da2.h5q3 noitprint;
class loc;
model torn = sst sst*loc / dist=poisson link=log ddfm=betwithin solution;
random intercept / subject=loc type=sp(exp)(lon lat);
nloptions tech=newrap;
covtest 'Random Int.' indep;
output out=h5q3out2 pred(ilink)=predicted lcl(ilink)=lower ucl(ilink)=upper residual(ilink)=Residual;
run;
proc glimmix data=da2.h5q3 noitprint;
class loc;
model torn = sst sst*loc / dist=poisson link=log ddfm=betwithin solution;
random sst / subject=loc type=sp(exp)(lon lat);
nloptions tech=newrap;
covtest 'Random Coef.' indep;
output out=h5q3out3 pred(ilink)=predicted lcl(ilink)=lower ucl(ilink)=upper residual(ilink)=Residual;
run;
proc glimmix data=da2.h5q3 noitprint;
class loc;
model torn = sst sst*loc / dist=poisson link=log ddfm=betwithin solution;
random intercept sst / subject=loc type=sp(exp)(lon lat);
nloptions tech=newrap;
covtest 'Random Int. & Coef.' indep;
output out=h5q3out4 pred(ilink)=predicted lcl(ilink)=lower ucl(ilink)=upper residual(ilink)=Residual;
run;


/*Processing output*/
proc sort data=h5q3out1;
by loc;
run;
data h5q3eval1;
set h5q3out1;
by loc;
keep loc torn predicted residual lat lon;
retain sumtorn sumpred sumres;
if first.loc then do;
sumtorn=0;
sumpred=0;
sumres=0;
end;
sumtorn+torn;
sumpred+predicted;
sumres+residual;
if last.loc then do;
torn=sumtorn;
predicted=sumpred;
residual=sumres;
output;
end;
run;
proc sort data=h5q3out2;
by loc;
run;
data h5q3eval2;
set h5q3out2;
by loc;
keep loc torn predicted residual lat lon;
retain sumtorn sumpred sumres;
if first.loc then do;
sumtorn=0;
sumpred=0;
sumres=0;
end;
sumtorn+torn;
sumpred+predicted;
sumres+residual;
if last.loc then do;
torn=sumtorn;
predicted=sumpred;
residual=sumres;
output;
end;
run;
proc sort data=h5q3out3;
by loc;
run;
data h5q3eval3;
set h5q3out3;
by loc;
keep loc torn predicted residual lat lon;
retain sumtorn sumpred sumres;
if first.loc then do;
sumtorn=0;
sumpred=0;
sumres=0;
end;
sumtorn+torn;
sumpred+predicted;
sumres+residual;
if last.loc then do;
torn=sumtorn;
predicted=sumpred;
residual=sumres;
output;
end;
run;
proc sort data=h5q3out4;
by loc;
run;
data h5q3eval4;
set h5q3out4;
by loc;
keep loc torn predicted residual lat lon;
retain sumtorn sumpred sumres;
if first.loc then do;
sumtorn=0;
sumpred=0;
sumres=0;
end;
sumtorn+torn;
sumpred+predicted;
sumres+residual;
if last.loc then do;
torn=sumtorn;
predicted=sumpred;
residual=sumres;
output;
end;
run;
data h5q3eval;
set h5q3eval1(in=a) h5q3eval2(in=b) h5q3eval3(in=c) h5q3eval4(in=d);
length model $23;
if a then do;
model='Independent';
end;
if b then do;
model='Random Int.';
end;
if c then do;
model='Random Coef.';
end;
if d then do;
model='Random Int. & Coef.';
end;
label torn='Actual Measurements';
run;



/*Evaluating models*/
proc sort data=h5q3eval;
by torn;
run;
proc sgpanel data=h5q3eval noautolegend;
panelby model/columns=2 rows=2 spacing=5;
scatter x=torn y=predicted/ datalabel=loc;
series x=torn y=torn;
KEYLEGEND "Observations" "Reference Line";
run;
proc sql;
title 'Model Comparation';
select model,sum(residual*residual) label='Model Type' as SSR label='Sum of Squared Residual'
from h5q3eval
group by model;
quit;

/*Plotting the profile*/
data panelplot2;
set h5q3out2;
length type $20;
keep loc t type resp;
t=t+1952;
type='measurement';
resp=torn;
output;
type='cluster-specific';
resp=predicted;
output;
type='lower bound';
resp=lower;
output;
type='upper bound';
resp=upper;
output;
run;
proc sgpanel data=panelplot2;
where loc le 4 and loc ge 1;
panelby loc/rows=2 columns=2 spacing=5;
vline t/response=resp group=type;
colaxis fitpolicy=thin alternate;
rowaxis alternate;
run;
proc sgpanel data=panelplot2;
where loc le 8 and loc ge 5;
panelby loc/rows=2 columns=2 spacing=5;
vline t/response=resp group=type;
colaxis fitpolicy=thin alternate;
rowaxis alternate;
run;
proc sgpanel data=panelplot2;
where loc le 12 and loc ge 9;
panelby loc/rows=2 columns=2 spacing=5;
vline t/response=resp group=type;
colaxis fitpolicy=thin alternate;
rowaxis alternate;
run;
proc sgpanel data=panelplot2;
where loc le 16 and loc ge 13;
panelby loc/rows=2 columns=2 spacing=5;
vline t/response=resp group=type;
colaxis fitpolicy=thin alternate;
rowaxis alternate;
run;
proc sgpanel data=panelplot2;
where loc le 20 and loc ge 17;
panelby loc/rows=2 columns=2 spacing=5;
vline t/response=resp group=type;
colaxis fitpolicy=thin alternate;
rowaxis alternate;
run;

\end{Sascode}


\Listing[store=class,
         caption={Regression Analysis}]{h5re}

\Graphic[store=class, scale=0.9,
         caption={Graphs for Regression Analysis}]{h5re}

\end{document}
